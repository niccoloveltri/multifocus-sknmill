\documentclass[runningheads]{llncs}
\providecommand{\event}{NCL'22} % Name of the event you are submitting to
%\usepackage{breakurl}             % Not needed if you use pdflatex only.
\usepackage{underscore}           % Only needed if you use pdflatex.
\usepackage{amsmath}
%\usepackage{amsthm}
\usepackage{amsfonts}
\usepackage{amssymb}
\usepackage{mathtools}
\usepackage{enumerate}
\usepackage{prftree}
%\usepackage{ntheorem}
%\usepackage{tikz-cd}
\usepackage{hyperref}
%\usepackage{float}
%\usepackage{graphicx}
%\usepackage{quiver}
\usepackage[all,cmtip]{xy}
\usepackage{proof}
\usepackage{dashbox}
%\usepackage[a4paper, total={6in, 8in}]{geometry}
%\usepackage{lscape}
%% \theorembodyfont{}
%\newtheorem{theorem}{Theorem}[section]
%\newtheorem{corollary}[theorem]{Corollary}
%\newtheorem{lemma}[theorem]{Lemma}
%\newtheorem{remark}[theorem]{Remark}
%\newtheorem{proposition}[theorem]{Proposition}
%\theoremstyle{definition}
%\newtheorem{defn}{Definition}[section]
%\newtheorem{example}{Example}[section]
%\newtheorem{defn}[definition]{Definition}
%\newtheorem{example}[definition]{Definition}
%\newtheorem*{proof}{Proof : }
\usepackage{cleveref}
\usepackage{bbm}
%\newtheorem{fact}[theorem]{Fact}
\usepackage{cancel}
\makeatletter
\newsavebox{\@brx}
\newcommand{\llangle}[1][]{\savebox{\@brx}{\(\m@th{#1\langle}\)}%
  \mathopen{\copy\@brx\kern-0.5\wd\@brx\usebox{\@brx}}}
\newcommand{\rrangle}[1][]{\savebox{\@brx}{\(\m@th{#1\rangle}\)}%
  \mathclose{\copy\@brx\kern-0.5\wd\@brx\usebox{\@brx}}}
\makeatother
\newcommand\dboxed[1]{\dbox{\ensuremath{#1}}}
\newcommand{\ldbc}{[\![}
\newcommand{\rdbc}{]\!]}
\newcommand{\tbar}{[\vec{x}/\vec{t}]}
\newcommand{\ltbar}{[\vec{x}, x/\vec{t}, x]}
\newcommand{\tl}{\otimes \mathsf{L}}
\newcommand{\tr}{\otimes \mathsf{R}}
\newcommand{\lright}{{\multimap}\mathsf{R}}
\newcommand{\lleft}{{\multimap}\mathsf{L}}
\newcommand{\pass}{\mathsf{pass}}
\newcommand{\M}{\mathsf{M}}
\newcommand{\unitl}{\mathsf{IL}}
\newcommand{\unitr}{\mathsf{IR}}
\newcommand{\otL}{\tl}
\newcommand{\otR}{\tr}
\newcommand{\lolliR}{\lright}
\newcommand{\lolliL}{\lleft}
\newcommand{\IL}{\unitl}
\newcommand{\IR}{\unitr}
\newcommand{\ax}{\mathsf{ax}}
\newcommand{\id}{\mathsf{id}}
\newcommand{\ot}{\otimes}
\newcommand{\lolli}{\multimap}
\newcommand{\illol}{\rotatebox[origin=c]{180}{$\multimap$}}
\newcommand{\I}{\mathsf{I}}
\newcommand{\msfL}{\mathsf{L}}
\newcommand{\defeq}{=_{\mathsf{df}}}
\newcommand{\comp}{\mathsf{comp}}
\newcommand{\RI}{\mathsf{RI}}
\newcommand{\LI}{\mathsf{LI}}
\newcommand{\Pass}{\mathsf{P}}
\newcommand{\F}{\mathsf{F}}
\newcommand{\xvdash}{\vdash^{x}}
\newcommand{\yvdash}{\vdash^{y}}

\newcommand{\otd}{\ot^D}
\newcommand{\lollid}{\;\textsuperscript{$D$}\!\!\lolli}

\newcommand{\highlight}[1]{\textcolor{blue}{#1}}

\newcommand{\proofbox}[1]{\begin{tabular}{l} #1 \end{tabular}}

\newcommand{\MILL}{\texttt{MILL}}
\newcommand{\NMILL}{\texttt{NMILL}}
\newcommand{\SkNMILL}{\texttt{SkNMILL}}
\newcommand{\FSkMCC}{\mathsf{Free}}

%%\newcommand{\tarmo}[1]{}
%%\newcommand{\niccolo}[1]{}
%%\newcommand{\cheng}[1]{}

\newcommand{\cheng}[1]{{\color{blue}\textbf{Cheng: }#1}}
\newcommand{\niccolo}[1]{{\color{red}\textbf{Niccol{\`o}: }#1}}
\newcommand{\tarmo}[1]{{\color{ForestGreen}\textbf{Tarmo: }#1}}

\usepackage{xcolor}

\newcommand\rfbox{\fcolorbox{red}{white}}

\newcommand{\up}{\Uparrow}
\newcommand{\dn}{\Downarrow}
\newcommand{\sw}{\mathsf{sw}}
\newcommand{\focL}{\mathsf{foc_L}}
\newcommand{\foc}{\mathsf{foc}}
\newcommand{\focR}{\mathsf{foc_R}}
\newcommand{\blurL}{\mathsf{blur_L}}
\newcommand{\blurR}{\mathsf{blur_R}}
\newcommand{\focLR}{\mathsf{foc_{M}}}
\newcommand{\blurLR}{\mathsf{blur_{M}}}
\newcommand{\focM}{\focLR}
\newcommand{\blurM}{\blurLR}
\renewcommand{\t}{\mathbbm{t}}
\renewcommand{\r}{\mathbbm{r}}
\renewcommand{\l}{\mathbbm{l}}
\newcommand{\tagL}{\mathsf{tag_L}}
\newcommand{\tagR}{\mathsf{tag_R}}
\newcommand{\tagM}{\mathsf{notag}}
\newcommand{\unfoc}{\mathsf{unfoc}}
%\newcommand{\blurR}{\mathsf{tag_R}}
%\newcommand{\blurM}{\mathsf{notag}}

\newcommand{\focus}{\mathsf{focus}}

\newcommand{\At}{\mathsf{At}}

\begin{document}

\title{Maximal Multifocusing for Skew Non-Commutative \MILL\thanks{This work was supported by the Estonian Research Council grant PSG749.}}

\author{Niccol{\`o} Veltri\inst{1}\orcidID{0000-0002-7230-3436}}
%
\authorrunning{N. Veltri}
%\def\titlerunning{} %Proof Theory of Skew Non-Commutative \MILL}
%\def\authorrunning{} %T. Uustalu, N. Veltri \& C.-S. Wan}
%\begin{document}
\institute{Tallinn University of Technology, Tallinn, Estonia \\
\email{niccolo@cs.ioc.ee}}
%
\maketitle
%
\begin{abstract}

\keywords{}  
\end{abstract}

\section{Introduction}\label{sec:intro}

\section{Sequent Calculus for \SkNMILL}\label{sec:seqcalc}

We recall the definition of the sequent calculus for \SkNMILL\ originally introduced in \cite{UVW:protsn}.
Formulae are inductively generated by the grammar $A,B ::= X \ | \ \I \ | \ A \ot B \ | \ A \lolli B$, where $X$ comes from a fixed set $\At$ of atomic formulae, $\I$ is a multiplicative unit, $\ot$ is a multiplicative tensor and $\lolli$ is a linear implication. Formulae $\I$ and $A \ot B$ are \emph{positive} while $A \lolli B$ is \emph{negative}. 

A sequent is a triple of the form $S \mid \Gamma \vdash A$, where the succedent $A$ is a single formula (as in \NMILL) and the antecedent is divided in two parts: an optional formula $S$, called \emph{stoup}, and an ordered list of formulae $\Gamma$, called \emph{context}. 
The metavariable $S$ always denotes a stoup, i.e. $S$ can be a single formula or empty, in which case we write $S = -$.

Derivations of a sequent $S \mid \Gamma \vdash A$ are inductively generated by the rules in \Cref{fig:seqcalc}.
%\begin{defn}
%  \emph{SkNMILL} is a sequent calculus system whose sequents are in the form of $S \mid \Gamma \vdash A$ coustructed with following rules
\begin{figure}[t]
  \[
  \def\arraystretch{2.5}
  \begin{array}{c}
    \infer[\ax]{X \mid \quad \vdash X}{}
    \qquad
    \infer[\pass]{- \mid A , \Gamma \vdash C}{A \mid \Gamma \vdash C}
    \\
    \infer[\unitl]{\I \mid \Gamma \vdash C}{- \mid \Gamma \vdash C}
    \qquad
    \infer[\tl]{A \ot B \mid \Gamma \vdash C}{A \mid B , \Gamma \vdash C}
    \qquad
    \infer[\lleft]{A \lolli B \mid \Gamma , \Delta \vdash C}{
      - \mid \Gamma \vdash A
      &
      B \mid \Delta \vdash C
    }
    \\
    \infer[\unitr]{- \mid \quad \vdash \I}{}
    \qquad
    \infer[\tr]{S \mid \Gamma , \Delta \vdash A \ot B}{
      S \mid \Gamma \vdash A
      &
      - \mid \Delta \vdash B
    }
    \qquad
    \infer[\lolliR]{S \mid \Gamma \vdash A \lolli B}{S \mid \Gamma , A \vdash B}
  \end{array}
  \]
\caption{Sequent calculus presentation of \SkNMILL.}
\label{fig:seqcalc}  
\end{figure}
There are a few differences from the sequent calculus of \NMILL:
1) left rules can only act on the formula in stoup position, not on formulae in context; 2) the right $\ot$-rule, when read bottom-up, forces the formula in the stoup (whenever it is present) to move to the stoup of the first premise, it cannot move to the antecedent of the second premise; 3) as in \NMILL, there are no structural rules of exchange, weakening and contraction, but there is a new structural rule $\pass$ which moves the leftmost formula in context to the stoup, whenever the latter is empty.

As in \NMILL\, rules $\IL$, $\otL$ and $\lolliR$ are invertible, while the other logical rules are not. Also non-invertible is the structural rule $\pass$.
Two forms of cut are admissible, depending of whether the cut formula belong to the stoup or the context of the second premise. A general axiom rule is derivable.
%\begin{proposition}
%The following rules are admissible:
\[
\small
    \infer[\mathsf{scut}]{S \mid \Gamma , \Delta \vdash C}{
      S \mid \Gamma \vdash A
      &
      A \mid \Delta \vdash C
    }
    \qquad
    \infer[\mathsf{ccut}]{S \mid \Delta_0 , \Gamma , \Delta_1 \vdash C}{
      - \mid \Gamma \vdash A
      &
      S \mid \Delta_0 , A , \Delta_1 \vdash C
    }
    \qquad
    \infer[\ax_A]{A \mid \quad \vdash A}{}
\]
%\end{proposition}

A stoup $S$ is called \emph{irreducible} if it is either empty, an atom or a negative formula. This means that the stoup formula cannot be further reduced using left invertible rules $\IL$ and $\otL$ in root-first proof search. Analogously, a succedent formula $A$ is irreducible when it is atomic or positive, so it cannot be reduced via the right invertible rule $\lolliR$.


%%The inference rules in (\ref{fig:seqcalc}) are reminiscent of the ones in the sequent calculus for \NMILL\ \cite{abrusci:noncommutative:1990}, but there are some crucial differences.
%%\begin{enumerate}
%%\item The left logical rules $\unitl$, $\tl$ and $\lleft$, read bottom-up, are only allowed to be applied on the formula in the stoup position. In particular, there is no general way to remove a unit $\I$ nor decompose a tensor $A \ot B$ if these formulae are located in the context and not in the stoup (we will see in (\ref{eq:lleft:gen}) that something can actually be done to deal with implications $A \lolli B$ in the context).
%%\item The right tensor rule $\tr$, read bottom-up, splits the antecedent of the conclusion between the two premises whereby the formula in the stoup, in case such a formula is present, has to be moved to the stoup of the first premise. In particular, the stoup formula of the conclusion cannot be moved to the antecedent of the second premise even if $\Gamma$ is chosen to be empty. 
%%\item The presence of the stoup implies a distinction between antecedents of the form $A \mid \Gamma$ and $- \mid \Gamma$. The structural rule $\pass$ (for `passivation`), read bottom-up, allows the moving of the leftmost formula in the context to the stoup position, whenever the stoup is initially empty.
%%\item The logical connectives of \NMILL\ typically include two ordered implications $\lolli$ and $\illol$, which are two variants of linear implication arising from the removal of the exchange rule from intuitionistic linear logic. In \SkNMILL\ only one of the ordered implications (the left implication $\lolli$) is present. It is currently not clear to us whether the inclusion of the second implication to our logic is a meaningful addition and whether it corresponds to some particular categorical notion.
%%\end{enumerate}
%%The restrictions in 1--4 are essential for precisely capturing all the features of skew monoidal closed categories and nothing more, as we discuss in Section \ref{sec:catsem}.
%%Notice also that, similarly to the case of \NMILL, all structural rules of exchange, contraction, and weakening are absent. We give names to derivations and we write $f : S \mid \Gamma \vdash A$ when $f$ is a particular derivation of the sequent $S \mid \Gamma \vdash A$.
%%
%%Examples of valid derivations in the sequent calculus, corresponding to structural laws $\lambda$, $\rho$ and $\alpha$ of skew monoidal closed categories (see Definition \ref{def:skewcat}), are given below. %which will help us  understanding the connection with skew monoidal closed categories.
%%%If we interpret morphism $A \Rightarrow C$ as a sequent $A \mid \quad \vdash C$, then we can see the natural transformations are derivable in this sequent calculus.
%%%For example, natural transformations $\lambda : \I \ot A \Rightarrow A , \rho : A \Rightarrow A \ot \I$, and $\alpha : (A \ot B) \ot C \Rightarrow A \ot (B \ot C)$ in skew monoidal closed categories are admissible in SkNMILL:
%%\begin{equation}\label{eq:lra}
%%  \begin{array}{c@{\;\quad}cc}
%%  (\lambda) & (\rho) & (\alpha) \\
%%  \infer[\tl]{\I \ot A \mid \quad \vdash A}{
%%    \infer[\unitl]{\I \mid A \vdash A}{
%%      \infer[\pass]{- \mid A \vdash A}{
%%        \infer[\ax]{A \mid \quad \vdash A}{}
%%      }
%%    }
%%  }
%%  &
%%  \infer[\tr]{A \mid \quad \vdash A \ot \I}{
%%    \infer[\ax]{A \mid \quad \vdash A}{}
%%    &
%%    \infer[\unitr]{- \mid \quad \vdash \I}{}
%%  }
%%  &
%%  \infer[\tl]{(A \ot B) \ot C \mid \quad \vdash A \ot (B \ot C)}{
%%    \infer[\tl]{A \ot B \mid C \vdash A \ot (B \ot C)}{
%%      \infer[\tr]{A \mid B , C \vdash A \ot (B \ot C)}{
%%        \infer[\ax]{A \mid \quad \vdash A}{}
%%        &
%%        \infer[\pass]{- \mid B , C \vdash B \ot C}{
%%          \infer[\tr]{B \mid C \vdash B \ot C}{
%%            \infer[\ax]{B \mid \quad \vdash B}{}
%%            &
%%            \infer[\pass]{- \mid C \vdash C}{
%%              \infer[\ax]{C \mid \quad \vdash C}{}
%%            }
%%          }
%%        }
%%      }
%%    }
%%  }
%%  \end{array}
%%\end{equation}
%%Examples of non-derivable sequents include the ``inverses'' of the conclusions in (\ref{eq:lra}), obtained by swapping the stoup formula with the succedent formula.
%%%Another important thing is that SkNMILL could not prove the inverse of any natural transformations above.
%%More precisely, the three sequents $X \mid ~ \vdash \I \ot X$, $X \ot \I \mid ~ \vdash X$ and $X \ot (Y \ot Z) \mid ~ \vdash (X \ot Y) \ot Z$ do not have any derivation. All possible attempts of constructing a valid derivation for each of them end in failure.
%%\begin{displaymath}
%%  \begin{array}{ccc}
%%  (\lambda^{-1}) & (\rho^{-1}) & (\alpha^{-1}) \\[6pt]
%%    \infer[\tr]{X \mid ~\vdash \I \ot X}{
%%      \deduce[??]{X \mid ~ \vdash \I}{
%%      }
%%      &
%%      \deduce[??]{- \mid ~ \vdash X}{
%%      }
%%    }
%%    &
%%    \infer[\tl]{X \ot \I \mid \quad \vdash X}{
%%      \deduce{X \mid \I \vdash X}{??}
%%    }
%%    &
%%    \infer[\tl]{X \ot (Y \ot Z) \mid ~\vdash (X \ot Y) \ot Z}{
%%      \deduce{X \mid Y \ot Z \vdash (X \ot Y) \ot Z}{??}
%%    } \\
%%    (\text{$\tr$ sends $X$ to 1st premise}) &
%%    (\text{$\unitl$ does not act on $\I$ in context}) &
%%    (\text{$\tl$ does not act on $\ot$ in context})
%%  \end{array}
%%\end{displaymath}
%%% We interpret $\rho^{-1}$ into $A \ot \I \mid \quad \vdash A$, then according to bottom-up proof search strategy, we first apply $\tl$ but we get stuck immediately.
%%% Therefore, $\rho^{-1}$ is not derivable in our sequent calculus.
%%% Other cases are similar.
%%% In categorical semantics section we will see more details about relationship between SkNMILL and free skew monoidal closed categories.
%%Analogously, the sequents $\I \lolli A \mid ~ \vdash A$ and $(A \ot B) \lolli C \mid ~ \vdash A \lolli (B \lolli C)$ are derivable, while generally their ``inverses'' are not. Also, a derivation of $A \mid ~ \vdash B$ always yields a derivation of $\I \mid ~ \vdash A \lolli B$, but there are $A$, $B$ such that $\I \mid ~ \vdash A \lolli B$ is derivable while  $A \mid ~ \vdash B$ is not (e.g., $A = X$, $B = \I \ot X$).

We consider an equivalence relation $\circeq$ on sets of derivations. This is the congruence generated by the pairs of derivations in \Cref{fig:circeq}. The fisrt group of equations consist of permutative conversions that, when oriented from left to write, move the application of an invertible rule under the application of a non-invertible rule. The last two equations are permutative conversions swapping the application of non-invertible rules $\otR/\pass$ and $\otR/\lolliL$.
\begin{figure}[t]
\begin{align*}\small
%%   \ax_{\I} &\circeq \unitl \text{ } (\unitr)
%%   \\
%%   \ax_{A \ot B} &\circeq \tl \text{ } (\tr \text{ } (\ax_{A} , \pass \text{ } \ax_{B}))
%%   \\
%%   \ax_{A \lolli B} &\circeq \lolliR \text{ } (\lleft \text{ } (\pass \text{ } \ax_{A}, \ax_{B} ))
%%   \\
  \tr \text{ } (\unitl \text{ } f, g) &\circeq \unitl \text{ } (\tr \text{ } (f , g)) &&(f : - \mid \Gamma \vdash A , g : - \mid \Delta \vdash B)
  \\
  \tr \text{ } (\tl \text{ } f, g) &\circeq \tl \text{ } (\tr \text{ } (f , g)) &&(f : A' \mid B' , \Gamma \vdash A , g : - \mid \Delta \vdash B)
  \\
%%   \tr \text{ } (\pass \text{ } f, g) &\circeq \pass \text{ } (\tr \text{ } (f, g)) &&f : A' \mid \Gamma \vdash A, g : - \mid \Delta \vdash B
%%   \\
%%   \tr \text{ } (\lleft \text{ } (f , g), h) & \circeq \lleft \text{ } (f, \tr \text{ } (g, h)) &&f: - \mid \Gamma \vdash A, g : B \mid \Delta \vdash C, h : - \mid \Lambda \vdash D
%%   \\
  \pass \text{ } (\lolliR \text{ } f) &\circeq \lolliR \text{ } (\pass \text{ } f) &&(f : A' \mid \Gamma , A \vdash B)
  \\
  \unitl \text{ } (\lolliR \text{ } f) &\circeq \lolliR \text{ } (\unitl \text{ } f) &&(f : - \mid \Gamma , A \vdash B)
  \\
  \tl \text{ } (\lolliR \text{ } f) &\circeq \lolliR \text{ } (\tl \text{ } f) &&(f : A \mid B , \Gamma , C \vdash D)
  \\
  \lleft \text{ } (f, \lolliR \text{ } g) &\circeq \lolliR \text{ } (\lleft \text{ } (f, g)) &&(f : - \mid \Gamma \vdash A', g : B' \mid \Delta , A \vdash B)
%\end{align*}
%\caption{Invertible/non-invertible permutative conversion for \SkNMILL.}
%\label{fig:circeq}
%\end{figure}
%\begin{figure}[t]
  %\begin{align*}\small
  \\[9pt]
  \tr \text{ } (\pass \text{ } f, g) &\circeq \pass \text{ } (\tr \text{ } (f, g)) &&(f : A' \mid \Gamma \vdash A, g : - \mid \Delta \vdash B)
  \\
  \tr \text{ } (\lleft \text{ } (f , g), h) & \circeq \lleft \text{ } (f, \tr \text{ } (g, h)) &&(f: - \mid \Gamma \vdash A, g : B \mid \Delta \vdash C, h : - \mid \Lambda \vdash D)
\end{align*}
\caption{Equivalence of derivations in \SkNMILL.}
\label{fig:circeq}
\end{figure}
The congruence $\circeq$ has been chosen to serve as the proof-theoretic counterpart of the equational theory of skew monoidal closed categories \cite{street:skew-closed:2013}.
In fact, there exists a \emph{syntactic} skew monoidal closed category $\FSkMCC(\At)$, which has formulae of $\SkNMILL$ as objects, and morphisms between formulae $A$ and $B$ are given by the set of derivations of $A \mid ~ \vdash B$ quotiented by the equivalence relation $\circeq_2$. The category $\FSkMCC(\At)$ is the \emph{free} skew monoidal closed category generated by the set of atomic formulae $\At$. We refer to \cite{UVW:protsn} for more details on categorical semantics.

In the rest of the paper we employ the following convention for naming formulae and stoups:
$P$ denotes a positive formula, $N$ a negative formula, $Q$ a positive or atomic formula, $M$ a negative or atomic formula, $S_M$ an irreducible stoup.





\section{A Multifocused Sequent Calculus}\label{sec:focus}

We present the multifocused sequent calculus for \SkNMILL, which draws strong inspiration from   the one given by Chaudhuri et al. for multiplicative-additive classical linear logic \cite{chaudhuri:canonical:2008}. Inference rules are given in \Cref{fig:multifocus}. As in the original formulation by Andreoli \cite{andreoli:logic:1992}, the (multi)focused calculus describes, in a declarative fashion, a root-first proof search strategy in the original sequent calculus.
\begin{figure}[t]
  \[
  \small
%  \def\arraystretch{2.5}
  \begin{array}{c}
    \multicolumn{1}{l}{\text{\textsc{Inversion/Passivation Phase}}} \\[4pt]
      \infer[\lolliR]{S \mid \Gamma \up A \lolli B}{S \mid \Gamma , A \up B}
    \qquad
    \infer[\otL]{A \ot B \mid \Gamma \up Q}{A \mid  B , \Gamma \up Q}
    \qquad
      \infer[\IL]{\I \mid \Gamma \up Q}{- \mid \Gamma \up Q}    
    \\[6pt]    
    \infer[\pass]{- \mid A , \Gamma \up Q}{
      A\mid \Gamma \up Q
    }
    \qquad
        \infer[\foc]{S_M \mid \Gamma \up Q}{
      \fbox{$S_M$}_{\, b} \mid \Gamma \dn \fbox{$Q$}_{\, b'}
    }
    \\[9pt]
    \multicolumn{1}{l}{\text{\textsc{Focusing Phase}}} \\[4pt]
    \infer[\lolliL]{\fbox{$A \lolli B$} \mid \Gamma , \Delta \dn \fbox{$Q$}_{\,b}}{
      - \mid \Gamma \up A
      &
      \fbox{$B$} \mid \Delta \dn \fbox{$Q$}_{\,b}
    }
    \qquad
        \infer[\blurL]{\fbox{$Q$} \mid \Gamma \dn \fbox{$A$}_{\,b}}{
      \dbox{$Q$} \mid \Gamma \dn \fbox{$A$}_{\,b}
      }    
    \\[8pt]
    \infer[\IR]{- \mid ~ \dn \fbox{$\I$}}{}
    \qquad
    \infer[\otR]{\dbox{$S_M$}_{\,b} \mid \Gamma , \Delta \dn \fbox{$A \ot B$}}{
      \dbox{$S_M$}_{\,b} \mid \Gamma \dn \fbox{$A$}
      &
      - \mid \Delta \up B
    }
    \qquad
        \infer[\blurR]{\dbox{$S_M$}_{\,b} \mid \Gamma \dn \fbox{$M$}}{
      \dbox{$S_M$}_{\,b} \mid \Gamma \dn \dbox{$M$}
    }
    \\[8pt]
    \infer[\ax]{\dbox{$X$} \mid ~ \dn \dbox{$X$}}{}
    \qquad
    \infer[\unfoc]{\dbox{$S$}_{\,b} \mid \Gamma \dn \dbox{$A$}_{\,b'}}{
      S \mid \Gamma \up A
      &
      b = 1 \supset S = P
      &
      b' = 1 \supset A = N
    }
  \end{array}
  \]
  \caption{Multifocused sequent calculus for \SkNMILL.} 
  \label{fig:multifocus}
\end{figure}

Proof search starts in the inversion/passivation phase $S \mid \Gamma \up A$. In this phase, invertible rules are repeatedly applyed until both the stoup formula (when present) and the succedent formula become irreducible. We have fixed an order on invertible rules and decided to apply $\lolliR$ before $\IL$/$\otR$, which is enforced by asking the succedent formula in the left invertible rules to be positive or atomic (so we use our notation $Q$). 
If the stoup is empty, we have the possibility of \emph{activating} the leftmost formula in context via $\pass$ and continue the inversion phase.

Proof search then progresses to the focusing phase via the rule $\foc$. At this point we can choose to focus on the stoup formula, the succedent formula, or both. Formulae on focus are surrounded by a \emph{solid} box {\small{\fbox{$A$}}}. To encapsulate all the focusing possibilities in one single rule $\foc$, we use notation {\small{$\fbox{$A$}_{\,b}$}}, with $b$ a Boolean value, to indicate whether formula $A$ is on focus or not: {\small{$\fbox{$A$}_{\,1} = \fbox{$A$}$}} and {\small{$\fbox{$A$}_{\,0} = A$}}. In other words, {\small{$\fbox{$A$}_{\,b}$}} denotes a formula which is possibly under focus. If the stoup formula is under focus, the left non-invertible rule $\lolliL$ is repeatedly applied until the stoup formula becomes positive or atomic ($Q$). When this happens, the left focus is \emph{blurred} via the rule $\blurL$. Blurred formulae are surrounded by a \emph{dashed} box ${\small{\dbox{$A$}}}$. A possibly-blurred formula is denoted ${\small{\dbox{$A$}_{\,b}}}$. If the succedent formula is under focus, the right non-invertible rule $\otR$ is repeatedly applied until the succedent becomes a unit, in which case the derivation is closed using $\IR$, or negative or atomic ($M$). When this happens, the right formula is blurred via $\blurR$.

When no formula is under focus anymore, and one or two formulae are blurred, we either close the derivation using $\ax$, or we \emph{unfocus} and proceed proof search in the inversion/passivation phase. In order to unfocus, formulae that where previously under focus, which are now blurred, must have switched their polarity, which is reflected in the side conditions of rule $\unfoc$: the stoup formula must be positive if it was under focus ($b=1$), while if the succedent formula was under focus ($b' = 1$) it must now be negative. The axiom rule is usable only if both stoup and succedent formulae were under focus.

Another peculiarity of the sequent calculus in \Cref{fig:multifocus}, when compared with other (multi)focused calculi appearing in the literature, e.g. the one for MALL in \cite{chaudhuri:canonical:2008}, is that during the application of non-invertible rules in the focusing phase, one of the premises always releases the focus. In rule $\otR$, the right premise releases the focus on the succedent formula, and similarly for the  first premise in rule $\lolliL$. Without the loss of focus in these premises, the multifocused sequent calculus would be unsound wrt. the calculus in \Cref{fig:seqcalc}, e.g. the sequent $X \mid Y  \ot Z \up X \ot (Y \ot Z)$ would not admit a derivation. This behaviour was already present in the focused sequent calculi for the $\ot$- and $(\I,\ot)$-fragments of the sequent calculus, originally studied by Zeilberger et al. \cite{zeilberger:semiassociative:19,uustalu:sequent:2021}.


\section{Maximal-Multifocusing Using Tags}\label{sec:maxfocus}

Four proof search phases:
\[
\begin{array}{rl}
  \text{Invertible/passivation :} &  S \mid \Gamma \up A \\
  \text{Left-focusing :} &  S \dn \Gamma \mid A \\
  \text{Right-focusing :} &  S \mid \Gamma \dn A \\
  \text{Multi-focusing :} &  S \dn \Gamma \dn A
\end{array}
\]
Sequents can also hold a tag $\ell$, meaning that we are the current goal arose from left focusing, or $r$, meaning that the current goal arose from right focusing. Metavariables $x,y$  denote tags, which can in particular be empty, i.e. not carrying a tag at all. A sequent can have at most one tag, so e.g. $S \mid^x \Gamma \up^y A$ means either $S \mid \Gamma \up A$ (no tags) or $S \mid^\ell \Gamma \up A$ (left tag is on) or $S \mid \Gamma \up^r A$ (right tag is on). In $\ell$-tagged contexts, formulae in context can also have tag $\ell$, corresponding to those formulae appearing in context by means of left invertible rules $\IL$, $\otL$ and passivation after termination of a left focusing phase. Similarly, in $r$-tagged contexts, formulae in context can have tag $r$, which implies that these formulae appeared in context by means of the right invertible rule $\lolliR$ after termination of a right focusing phase. Contexts of the form $\Gamma^{\bcancel{\ell}}$ or $\Gamma^{\bcancel{r}}$ are the same as $\Gamma$ but with all tags removed.

Focused derivations are inductively generated by the following rules:
\begin{equation}\label{fig:max-multifocus}\small
%  \def\arraystretch{2.5}
  \begin{array}{c}
    \multicolumn{1}{l}{\text{\textsc{Invertible/Passivation Phase}}} \\[4pt]
      \infer[\lolliR]{S \mid \Gamma \up^{\r} A \lolli B}{S \mid \Gamma , A^{\r} \up^{\r} B}
    \qquad
%    \infer[\sw]{S \mid \Gamma \up^{r} P}{S \up \Gamma \mid^{r} P}
%  \end{displaymath}
    %  \begin{displaymath}
%    \\[6pt]
    \infer[\otL]{A \ot B \mid^\l \Gamma \up Q}{A \mid^\l  B^{\l} , \Gamma \up Q}
    \qquad
      \infer[\IL]{\I \mid^\l \Gamma \up Q}{- \mid^\l \Gamma \up Q}    
    \qquad
%    \infer[\sw]{T \up \Gamma \xvdash_{\LI} P}{T \mid \Gamma \xvdash_{\Pass} P}
%  \end{displaymath}
    %  \begin{displaymath}
    \\[6pt]    
    \infer[\pass_\bullet]{- \mid A^{\bullet} , \Gamma \up^{\bullet} Q }{
      A\mid \Gamma^{\circ} \up Q
    }
    \qquad
    \infer[\pass]{- \mid^{\l} A , \Gamma \up Q }{
      A^\circ \mid^{\l} \Gamma \up Q
    }
    \qquad
        \infer[\foc]{S_M \mid^{\l} \Gamma \up^{\r} Q}{
      \fbox{$S_M$}^{\, ?} \mid^\l \Gamma \dn^{\r} \fbox{$Q$}^{\, ?}
    }
%    \\[6pt]
%    %
%    \infer[\focL]{M \mid^{x} \Gamma \up^{y} Q}{
%      M \dn \Gamma^{\bcancel{\ell}} \mid^{y} Q
%    }
%    \qquad
%    \infer[\focR]{S_M \mid^{x} \Gamma \up^{y} P}{
%      S_M \mid^{x} \Gamma^{\bcancel{r}} \dn P
%    }
%    \qquad
%    \infer[\focLR]{N \mid^{x} \Gamma \up^{y} P}{
%      N \dn^{x} \Gamma \dn^y P
%    }
%  \end{displaymath}
    %  \begin{displaymath}
    \\[9pt]
%    \hline
    \multicolumn{1}{l}{\text{\textsc{Focusing Phase}}} \\[4pt]
    \infer[\lolliL]{\fbox{$A \lolli B$} \mid^\l \Gamma , \Delta \dn^\r \fbox{$Q$}^{\,?}}{
      - \mid \Gamma^\circ \up A
      &
      \fbox{$B$} \mid^\l \Delta \dn^\r \fbox{$Q$}^{\,?}
    }
    \qquad
    \infer[\lolliL_\bullet]{\fbox{$A \lolli B$} \mid \Gamma , \Delta \dn^\bullet \fbox{$Q$}^{\,?}}{
      - \mid \Gamma^{\circ} \up A
      &
      \fbox{$B$} \mid \Delta^{\circ} \dn \fbox{$Q$}^{\,?}
      &
      \bullet \in \Gamma
    }
%%    \infer[\lleft^+]{\vec{A} \lolli B \dn \vec{\Gamma} , \Delta \mid^r Q}{
%%      \{- \mid \Gamma_j^{\bcancel{r}} \up A_j\}_{j>0}
%%      &
%%      B \dn \Delta^{\bcancel{r}} \mid Q
%%      &
%%      r \in \vec{\Gamma}
    %%    }
        \\[6pt]
%        \hline
%    \multicolumn{1}{l}{\text{\textsc{Right-focusing}}} \\
    \infer[\otR]{S_M \mid^\l \Gamma , \Delta \dn \fbox{$A \ot B$}}{
      S_M \mid^\l \Gamma \dn \fbox{$A$}
      &
      - \mid \Delta^\circ \up B
    }
    \qquad
    \infer[\otR_\bullet]{S_M \mid^{\bullet} \Gamma , \Delta \dn \fbox{$A \ot B$}}{
      S_M \mid \Gamma^{\circ} \dn \fbox{$A$}
      &
      - \mid \Delta^{\circ} \up B
      &
      \bullet \in \Delta
    }
    \\[6pt]
    \infer[\IR]{- \mid ~ \dn \fbox{$\I$}}{}
    \qquad
        \infer[\blurL]{\fbox{$Q$} \mid^\l \Gamma \dn \fbox{$A$}^{\,?}}{
      \dbox{$Q$} \mid^\l \Gamma \dn \fbox{$A$}^{\,?}
      }    
    \qquad
        \infer[\blurR]{\dbox{$S_M$}^{\,?} \mid \Gamma \dn \fbox{$M$}}{
      \dbox{$S_M$}^{\,?} \mid \Gamma \dn \dbox{$M$}
    }
    \\[6pt]
    \infer[\ax]{\dbox{$X$} \mid ~ \dn \dbox{$X$}}{}
    \qquad
    \infer[\tagL]{\dbox{$P$} \mid \Gamma \dn A}{
      P \mid^\bullet \Gamma \up A}
    \qquad
    \infer[\tagR]{S_M \mid \Gamma \dn \dbox{$N$}}{
      S_M \mid \Gamma \up^\bullet N }
    \qquad
    \infer[\tagM]{\dbox{$P$} \mid \Gamma \dn \dbox{$N$}}{
    P \mid \Gamma \up N}
%    \quad
%%    \infer[\tr^+]{S_M \mid^{\ell} \Gamma , \vec{\Delta} \dn A \ot \vec{B}}{
%%      S_M \mid \Gamma^{\bcancel{\ell}} \dn A
%%      &
%%      \{ - \mid \Delta_i^{\bcancel{\ell}} \up B_i\}_{i>0}
%%      &
%%      \ell \in \vec{\Delta}
%%    }
%%    \\[6pt]
%%    \infer[\lleft\!\tr^+]{\vec{A} \lolli C \dn^{\ell} \vec{\Gamma}, \Lambda, \vec{\Delta} \dn^r D \ot \vec{B}}{
%%      \{- \mid \Gamma_j^{\bcancel{\ell}\bcancel{r}} \up A_j\}_{j>0} &
%%      C \mid \Lambda^{\bcancel{\ell}\bcancel{r}} \up D
%%      &
%%      \{ - \mid \Delta_i^{\bcancel{\ell}\bcancel{r}} \up B_i\}_{i>0}
%%      &
%%      \ell \in \vec{\Delta}
%%      &
%%      r \in \vec{\Gamma}
%%    }
%    \\[6pt]
%    \hline
%    \multicolumn{1}{l}{\text{\textsc{Multi-focusing}}} \\
%    \infer[\lleft_\M]{A \lolli B \dn^x \Gamma , \Delta \dn^y C}{
%      ~ \mid \Gamma^{\bcancel{\ell}} \up A
%      &
%      B \dn^x \Delta \dn^y C
%    }
%    \qquad
%    \infer[\lleft^r_\M]{A \lolli B \dn^x \Gamma , \Delta \dn^r C}{
%      ~ \mid \Gamma^{\bcancel{\ell}\bcancel{r}} \up A
%      &
%      B \dn^x \Delta^{\bcancel{r}} \dn C
%      &
%      r \in \Gamma
%    }
%    \\[6pt]
%    \infer[\tr_\M]{Q \dn^{x} \Gamma , \Delta \dn A \ot B}{
%      Q \dn^x \Gamma \dn A
%      &
%      - \mid \Delta \up B
%    }
%    \qquad
%    \infer[\tr^\ell_\M]{Q \dn^{\ell} \Gamma , \Delta \dn A \ot B}{
%      Q \dn \Gamma^{\bcancel{\ell}} \dn A
%      &
%      - \mid \Delta^{\bcancel{\ell}} \up B
%      &
%      \ell \in \Delta
%    }
%    \qquad
%        \infer[\blurLR]{Q \dn \Gamma \dn M}{
%      Q \mid \Gamma \up M
%      }
%    \hspace{4.1pt}
%  \end{displaymath}
%  \begin{displaymath}
  \end{array}
\end{equation}
%\qquad
%R = \I,M
%\]
%%\niccolo{What about these changes?
%%
%%
%%  - In right-focusing phase, condition $\ell \in \vec{\Delta}$ not always necessary. E.g. try to prove $\I \lolli \I | ~ \up \I \ot \I$
%%
%%%  - In multifocusing phase, replace $R$ by $M$ and add possibility of doing $\IR$
%%  }

The focused calculus in (\ref{fig:max-multifocus}) is sound wrt. the sequent calculus in (\ref{fig:seqcalc}). It
is also complete, which follows from Propositions \ref{prop:adm1}
and \ref{prop:adm2} below.
\begin{proposition}\label{prop:adm1}
  The following rules are admissible:
  \[
      \infer[\unitl_\up]{\I \mid \Gamma \up C}{- \mid \Gamma \up C}
    \qquad
    \infer[\tl_\up]{A \ot B \mid\Gamma \up C}{A \mid B , \Gamma \up C}
    \qquad
%    \infer[\sw]{T \up \Gamma \xvdash_{\LI} P}{T \mid \Gamma \xvdash_{\Pass} P}
%  \end{displaymath}
    %  \begin{displaymath}
    \infer[\pass_\up]{- \mid A , \Gamma \up C }{
      A\mid \Gamma \up C
    }
    \]
\end{proposition}
\begin{proof}
  Induction on the premise of each rule.
\end{proof}

%%Define the relation $\Gamma\mid A \rightsquigarrow_{\mathsf{R}} \Delta \mid B$ via the rules:
%%\[
%%\infer{\Gamma\mid A \rightsquigarrow_{\mathsf{R}} \Gamma\mid A}{}
%%\qquad
%%\infer{\Gamma\mid A \rightsquigarrow_{\mathsf{R}} \Delta \mid B \lolli C}{
%%  \Gamma\mid A \rightsquigarrow_{\mathsf{R}} \Delta,B \mid C
%%}
%%\]
\begin{lemma}\label{lem:otR}
  The following rules are admissible:
  \[
%  \infer[\tr_{\up Q}^+]{S \mid^{\ell} \Gamma , \vec{\Delta} \up Q \ot \vec{B}}{
%      S \mid^{\ell} \Gamma \up Q
%      &
%      \{ - \mid \Delta_i \up B_i\}_{i>0}
%  }
%  \qquad
  \infer[\tr_{\up P}^+]{S \mid^x \Gamma , \vec{\Delta} \up P \ot \vec{B}}{
      \deduce{S \mid^x \Gamma \up P}{f}
      &
      \deduce{\{ - \mid \Delta_i \up B_i\}_{i>0}}{g_i}
%      & \Gamma' \not= -
  }
  \qquad
  \infer[\tr_{\up N}^+]{S \mid \Gamma , \vec{\Delta} \up (\Gamma' \lolli Q) \ot \vec{B}}{
      \deduce{S \mid \Gamma,\Gamma' \up Q}{f}
      &
      \deduce{\{ - \mid \Delta_i \up B_i\}_{i>0}}{g_i}
%      & \Gamma' \mid Q \rightsquigarrow_{\mathsf{R}} ~ \mid M
  }
  \]
  %In $\tr_{\up P}^+$ we additionally require that, when $S = S_M$ and
  %$\ell = \bullet$ then $\bullet \in \Gamma$.
\end{lemma}
\begin{proof}[Proof of $\tr_{\up P}^+$]%\renewcommand{\qedsymbol}{}
%  We only discuss $\tr_\up^+$, the proof of $\lleft_{\up}^+$ is
%  analogous (kinda specular).
  By induction on $f$, the challenging cases are the focusing rules.
%  \begin{itemize}
\\\fbox{Case $f = \focL(f')$} with $f' : M \dn \Gamma \mid P$ and 
%\\    If $f' = \ax$, so $M = Q = X$ and $\Gamma = -$,
%then 
%    \[\small
%    \proofbox{
%      \infer[\tr_{\up P}^+]{X \mid \vec{\Delta} \up X \ot \vec{B}}{
%        \infer[\focL]{X \mid ~ \up X}{
%          \infer[\ax]{X \dn ~ \mid X}{}
%      }
%      &
%      \deduce{\{ - \mid \Delta_i \up B_i\}}{g_i}
%      }
%    }
%    =
%    \proofbox{
%      \infer[\focR]{X \mid \vec{\Delta} \up X \ot \vec{B}}{
%        \infer[\otR^+]{X \mid \vec{\Delta} \dn X \ot \vec{B}}{
%          \infer[\ax]{X \mid ~ \dn X}{} &
%      &
%      \deduce{\{ - \mid \Delta_i \up B_i\}}{g_i}          
%        }
%      }
%    }
%    \]
%    \\If $f' = \lolliL(\{f_j\},\ax)$ then
%    \[\small
%    \proofbox{
%      \infer[\tr_{\up P}^+]{\vec{A} \lolli X \mid \vec{\Gamma},\vec{\Delta} \up X \ot \vec{B}}{
%        \infer[\focL]{\vec{A} \lolli X \mid \vec{\Gamma} \up X}{
%          \infer[\lolliL^+]{\vec{A} \lolli X \dn \vec{\Gamma} \mid X}{
%          \deduce{\{ - \mid \Gamma_j \up A_j\}}{f_j}
%          &
%          \infer[\ax]{X \dn ~ \mid X}{}
%          }
%        }
%      &
%      \deduce{\{ - \mid \Delta_i \up B_i\}}{g_i}
%      }
%    }
%    =
%    \proofbox{
%      \infer[\focLR]{\vec{A} \lolli X \mid \vec{\Gamma},\vec{\Delta} \up X \ot \vec{B}}{
%        \infer[\lolliL\otR^+]{\vec{A} \lolli X \dn \vec{\Gamma},\vec{\Delta} \dn X \ot \vec{B}}{
%          \deduce{\{ - \mid \Gamma_j \up A_j\}}{f_j}
%          &
%          \infer[\blurLR]{X \dn ~ \dn X}{
%            \infer[\focL]{X \mid ~ \up X}{
%              \infer[\ax]{X \dn ~ \mid X}{}
%            }
%          }          
%          &
%          \deduce{\{ - \mid \Delta_i \up B_i\}}{g_i}
%        }
%      }
%    }
%    \]
    $f' = \lolliL^+(\{f_j\},\blurL(h))$:
    \[\footnotesize%\small
    \proofbox{
      \infer[\tr_{\up P}^+]{\vec{A} \lolli P' \mid^x \vec{\Gamma},\Lambda,\vec{\Delta} \up P \ot \vec{B}}{
        \infer[\focL]{\vec{A} \lolli P' \mid^x \vec{\Gamma},\Lambda \up P}{
          \infer[\lolliL^+]{\vec{A} \lolli P' \dn \vec{\Gamma},\Lambda \mid P}{
          \deduce{\{ - \mid \Gamma_j \up A_j\}}{f_j}
          &
          \infer[\blurL]{P' \dn \Lambda \mid P}{
            \deduce{P' \mid^\ell \Lambda \up P}{h}
          }
          }
        }
      &
      \deduce{\{ - \mid \Delta_i \up B_i\}}{g_i}
      }
    }
    \hspace{-.6cm}
    =
    \proofbox{
      \infer[\focL]{\vec{A} \lolli P' \mid^x \vec{\Gamma},\Lambda,\vec{\Delta} \up P \ot \vec{B}}{
        \infer[\lolliL^+]{\vec{A} \lolli P' \dn \vec{\Gamma}, \Lambda, \vec{\Delta} \mid P \ot \vec{B}}{
          \deduce{\{ - \mid \Gamma_j \up A_j\}}{f_j}
          &
          \infer[\blurL]{P' \dn \Lambda, \vec{\Delta} \mid P \ot \vec{B}}{
            \infer[\otR^+_{\up P}]{P' \mid^\ell \Lambda, \vec{\Delta} \up P \ot \vec{B}}{
              \deduce{P' \mid^\ell \Lambda \up P}{h}
      &
      \deduce{\{ - \mid \Delta_i \up B_i\}}{g_i}
        }
      }
        }
      }
    }
    \]
    \fbox{Case $f = \focLR(f')$} with $f' : N \dn \Gamma \dn P$ and $f' = \lolliL\otR^+(\{f_j\},\blurLR(k),\{h_l\})$:
    \[\footnotesize \hspace{-2cm}
    \proofbox{
      \infer[\tr^+_{\up P}]{\vec{A} \lolli Q \mid^x \vec{\Gamma},\Lambda,\vec{\Omega},\vec{\Delta} \up (M \ot \vec{C}) \ot \vec{B}}{
        \infer[\focLR]{\vec{A} \lolli Q \mid^x \vec{\Gamma}, \Lambda, \vec{\Omega}, \up M \ot \vec{C}}{
          \infer[\lolliL\otR^+]{\vec{A} \lolli Q \dn^x \vec{\Gamma}, \Lambda, \vec{\Omega}, \dn M \ot \vec{C}}{
            \deduce{\{ - \mid \Gamma_j^{\bcancel{\ell}} \up A_j\}}{f_j}
            &
            \infer[\blurLR]{Q \dn \Lambda^{\bcancel{\ell}} \dn M}{
              \deduce{Q \mid \Lambda^{\bcancel{\ell}} \up M}{k}
            }
            &
            \deduce{\{ - \mid \Omega_l^{\bcancel{\ell}} \up C_l\}}{h_l}
            &
            x \in \vec{\Omega}
          }
      }
          \deduce{\{ - \mid \Delta_i \up B_i\}}{g_i}            
      }
    }
    \]
    \[\footnotesize \hspace{2cm}
    =
    \proofbox{
      \infer[\focLR]{\vec{A} \lolli Q \mid^x \vec{\Gamma}, \Lambda, \vec{\Omega}, \vec{\Delta} \up (M \ot \vec{C}) \ot \vec{B}}{
        \infer[\lolliL\otR^+]{\vec{A} \lolli Q \dn^x \vec{\Gamma}, \Lambda, \vec{\Omega}, \vec{\Delta} \dn (M \ot \vec{C}) \ot \vec{B}}{
            \deduce{\{ - \mid \Gamma_j^{\bcancel{\ell}} \up A_j\}}{f_j}
            &
            \infer[\blurLR]{Q \dn \Lambda^{\bcancel{\ell}} \dn M}{
              \deduce{Q \mid \Lambda^{\bcancel{\ell}} \up M}{k}
            }
            &
            \deduce{\{ - \mid \Omega_l^{\bcancel{\ell}} \up C_l\}}{h_l} \cup
            \deduce{\{ - \mid \Delta_i \up B_i\}}{g_i}
            &
            x \in \vec{\Omega}
        }
      }
    }
    \]
    \fbox{Case $f = \focR(f')$} with $f' : S_M \mid \Gamma \dn P$:
    \[\footnotesize
    \proofbox{
      \infer[\tr_{\up P}^+]{S_M \mid^x \Gamma , \vec{\Delta} \up P \ot \vec{B}}{
        \infer[\focR]{S_M \mid^x \Gamma \up P}{
          \deduce{S_M \mid^x \Gamma \dn P}{f'}
        }
        &
        \deduce{\{ - \mid \Delta_i \up B_i\}}{g_i}
      }
    }
    =
    \proofbox{
      \infer[\focR]{S_M \mid^x \Gamma , \vec{\Delta} \up P \ot \vec{B}}{
        \infer[\otR^+]{S_M \mid^x \Gamma , \vec{\Delta} \dn P \ot \vec{B}}{
          \deduce{S_M \mid^x \Gamma \dn P}{f'}
          &
          \deduce{\{ - \mid \Delta_i \up B_i\}}{g_i}
        }
      }
    }
    \]
\end{proof}
\begin{proof}[Proof of $\tr_{\up N}^+$]
%  We only discuss $\tr_\up^+$, the proof of $\lleft_{\up}^+$ is
%  analogous (kinda specular).
  By induction on $f$, the challenging cases
  are $\pass$ and the focusing rules.
  \\
  \fbox{Case $f = \pass(f')$} with $f' : C \mid \Omega \up Q$ and
  $C,\Omega = \Gamma,\Gamma'$:
  \\If $A'$ is in the head of $\Gamma$,
    then we simply permute $\tr^+_\up$ with $\pass$. If $\Gamma = -$,
    then
    \[\footnotesize
    \proofbox{
      \infer[\tr_{\up N}^+]{- \mid \vec{\Delta} \up ((C, \Omega) \lolli Q) \ot \vec{B}}{
      \infer[\pass]{- \mid C,\Omega \up Q}{
        \deduce{C \mid \Omega \up Q}{f'}
      }
      &
      \deduce{\{ - \mid \Delta_i \up B_i\}}{g_i}
      }
    }
    =
    \proofbox{
      \infer[\focR]{- \mid \vec{\Delta} \up ((C ,\Omega) \lolli Q) \ot \vec{B}}{
        \infer[\tr^+]{- \mid \vec{\Delta} \dn ((C,\Omega) \lolli Q) \ot \vec{B}}{
          \infer[\blurR]{- \mid ~ \dn (C,\Omega) \lolli Q}{
            \infer[\lolliR^+]{- \mid ~ \up^r (C,\Omega) \lolli Q}{
              \infer[\pass]{ - \mid C^r,\Omega^r \up^r Q}{
                \deduce{C \mid \Omega \up Q}{f'}
              }
            }
          }
          &
          \deduce{\{ - \mid \Delta_i \up B_i\}}{g_i}
      }
      }
    }
    \]
    \fbox{Case $f = \focL(f')$} with $f' : M' \dn \Gamma,\Gamma' \mid Q$:
\\    If $f' = \ax$, so $M' = Q = X$ and $\Gamma = \Gamma' = -$,
then 
   \[\footnotesize
   \proofbox{
     \infer[\tr_{\up N}^+]{X \mid \vec{\Delta} \up X \ot \vec{B}}{
       \infer[\focL]{X \mid ~ \up X}{
         \infer[\ax]{X \dn ~ \mid X}{}
     }
     &
     \deduce{\{ - \mid \Delta_i \up B_i\}}{g_i}
     }
   }
   =
   \proofbox{
     \infer[\focR]{X \mid \vec{\Delta} \up X \ot \vec{B}}{
       \infer[\otR^+]{X \mid \vec{\Delta} \dn X \ot \vec{B}}{
         \infer[\blurR]{X \mid ~ \dn X}{
           \infer[\focL]{X \mid ~ \up^r X}{
             \infer[\ax]{X \dn ~ \up^r X}{}
           }
         } &
     &
     \deduce{\{ - \mid \Delta_i \up B_i\}}{g_i}          
       }
     }
   }
   \]
    If $f' = \lolliL^+(\{f_j\},h)$ and the $f_j$ do not use resources from $\Gamma'$:
    \[\footnotesize
    \proofbox{
      \infer[\tr_{\up N}^+]{\vec{A} \lolli Q' \mid \vec{\Gamma},\Lambda,\vec{\Delta} \up M \ot \vec{B}}{
        \infer[\focL]{\vec{A} \lolli Q' \mid \vec{\Gamma},\Lambda,\Gamma' \up Q}{
          \infer[\lolliL^+]{\vec{A} \lolli Q' \dn \vec{\Gamma},\Lambda,\Gamma' \mid Q}{
          \deduce{\{ - \mid \Gamma_j \up A_j\}}{f_j}
          &
          \deduce{Q' \dn \Lambda,\Gamma' \mid Q}{h}
          }
        }
      &
      \deduce{\{ - \mid \Delta_i \up B_i\}}{g_i}
      }
    }
    \hspace{-1.257 cm}
    =
    \proofbox{
      \infer[\focLR]{\vec{A} \lolli Q' \mid \vec{\Gamma},\Lambda,\vec{\Delta} \up M \ot \vec{B}}{
        \infer[\lolliL\otR^+]{\vec{A} \lolli Q' \dn \vec{\Gamma}, \Lambda, \vec{\Delta} \dn M \ot \vec{B}}{
          \deduce{\{ - \mid \Gamma_j \up A_j\}}{f_j}
          &
          \infer[\lolliR^{\star}]{Q' \mid \Lambda \up M}{
            \deduce{Q' \mid \Lambda,\Gamma' \up Q}{h'}
          }
          &
          \deduce{\{ - \mid \Delta_i \up B_i\}}{g_i}
        }
      }
    }
    \]
    where $h'$ is derivable from $h$.
    \\If $f' = \lolliL^+(\{f_j\},h)$ and the $f_j$ do use resources from $\Gamma'$, so $\Gamma = \vec{\Gamma_0}$ and $\Gamma' = \vec{\Gamma_1},\Lambda$ for $\vec{\Gamma_1} \not= -$:
    \[\footnotesize
    \proofbox{
      \infer[\tr_{\up N}^+]{\vec{A} \lolli Q' \mid \vec{\Gamma_0},\vec{\Delta} \up M \ot \vec{B}}{
        \infer[\focL]{\vec{A} \lolli Q' \mid \vec{\Gamma_0},\vec{\Gamma_1},\Lambda \up Q}{
          \infer[\lolliL^+]{\vec{A} \lolli Q' \dn \vec{\Gamma_0},\vec{\Gamma_1},\Lambda \mid Q}{
          \deduce{\{ - \mid \Gamma_j \up A_j\}}{f_j}
          &
          \deduce{Q' \dn \Lambda \mid Q}{h}
          }
        }
      &
      \deduce{\{ - \mid \Delta_i \up B_i\}}{g_i}
      }
    }
    \hspace{-.6cm}
    =
   \proofbox{
     \infer[\focR]{\vec{A} \lolli Q' \mid \vec{\Gamma_0},\vec{\Delta} \up M \ot \vec{B}}{
       \infer[\otR^+]{\vec{A} \lolli Q' \mid \vec{\Gamma_0},\vec{\Delta} \dn M \ot \vec{B}}{
         \infer[\blurR]{\vec{A} \lolli Q' \mid \vec{\Gamma_0} \dn M}{
           \infer[\lolliR^+]{\vec{A} \lolli Q' \mid \vec{\Gamma_0} \up^r M}{
             \infer[\focL]{\vec{A} \lolli Q' \mid \vec{\Gamma_0}, \vec{\Gamma_1}^r,\Lambda^r \up^r Q}{
               \infer[\lolliL^+]{\vec{A} \lolli Q' \dn \vec{\Gamma_0}, \vec{\Gamma_1}^r, \Lambda^r \mid^r Q}{
                 \deduce{\{ - \mid \Gamma_j \up A_j\}}{f_j}
                 &
                 \deduce{Q' \dn \Lambda \mid Q
               }{h}
             }
           }
         }
       }
         &
         \deduce{\{ - \mid \Delta_i \up B_i\}}{g_i}
     }
     }
   }
    \]
%    \\If $f' = \lolliL(\{f_j\},\ax)$ then
%    \[\small
%    \proofbox{
%      \infer[\tr_{\up P}^+]{\vec{A} \lolli X \mid \vec{\Gamma},\vec{\Delta} \up X \ot \vec{B}}{
%        \infer[\focL]{\vec{A} \lolli X \mid \vec{\Gamma} \up X}{
%          \infer[\lolliL^+]{\vec{A} \lolli X \dn \vec{\Gamma} \mid X}{
%          \deduce{\{ - \mid \Gamma_j \up A_j\}}{f_j}
%          &
%          \infer[\ax]{X \dn ~ \mid X}{}
%          }
%        }
%      &
%      \deduce{\{ - \mid \Delta_i \up B_i\}}{g_i}
%      }
%    }
%    =
%    \proofbox{
%      \infer[\focLR]{\vec{A} \lolli X \mid \vec{\Gamma},\vec{\Delta} \up X \ot \vec{B}}{
%        \infer[\lolliL\otR^+]{\vec{A} \lolli X \dn \vec{\Gamma},\vec{\Delta} \dn X \ot \vec{B}}{
%          \deduce{\{ - \mid \Gamma_j \up A_j\}}{f_j}
%          &
%          \infer[\blurLR]{X \dn ~ \dn X}{
%            \infer[\focL]{X \mid ~ \up X}{
%              \infer[\ax]{X \dn ~ \mid X}{}
%            }
%          }          
%          &
%          \deduce{\{ - \mid \Delta_i \up B_i\}}{g_i}
%        }
%      }
%    }
%    \]
    \fbox{Case $f = \focLR(f')$} with $f' : N \dn \Gamma,\Gamma' \dn P$.
    \\If $f' = \lolliL\otR^+(\{f_j\},\blurLR(k),\{h_l\})$ and the $f_i$ do not use resources in $\Gamma'$, so $\Gamma,\Gamma' = \vec{\Gamma},\Lambda,\vec{\Omega}$ and $\Gamma = \vec{\Gamma},\Lambda'$:
    \[\footnotesize \hspace{-2cm}
    \proofbox{
      \infer[\tr^+_{\up N}]{\vec{A} \lolli Q \mid \Gamma,\vec{\Delta} \up M \ot \vec{B}}{
        \infer[\focLR]{\vec{A} \lolli Q \mid \vec{\Gamma}, \Lambda, \vec{\Omega}, \up M' \ot \vec{C}}{
          \infer[\lolliL\otR^+]{\vec{A} \lolli Q \dn \vec{\Gamma}, \Lambda,\vec{\Omega}, \dn M' \ot \vec{C}}{
            \deduce{\{ - \mid \Gamma_j\up A_j\}}{f_j}
            &
            \infer[\blurLR]{Q \dn \Lambda \dn M'}{
              \deduce{Q \mid \Lambda \up M'}{k}
            }
            &
            \deduce{\{ - \mid \Omega_l \up C_l\}}{h_l}
          }
      }
          \deduce{\{ - \mid \Delta_i \up B_i\}}{g_i}            
      }
    }
    \]
    \[\footnotesize \hspace{2cm}
    =
    \proofbox{
      \infer[\focLR]{\vec{A} \lolli Q \mid \Gamma, \vec{\Delta} \up M \ot \vec{B}}{
        \infer[\lolliL\otR^+]{\vec{A} \lolli Q \dn \Gamma,\vec{\Delta} \dn M \ot \vec{B}}{
            \deduce{\{ - \mid \Gamma_j \up A_j\}}{f_j}
            &
            \infer[\blurLR]{Q \dn \Lambda' \dn M}{
              \infer[\lolliR^+]{Q \mid \Lambda' \up M}{
                \infer[\tr^+_{\up N}]{Q \mid \Lambda,\vec{\Omega} \up M' \ot \vec{C}}{
                  \deduce{Q \mid \Lambda \up M'}{k}           
                  &
                  \deduce{\{ - \mid \Omega_l \up C_l\}}{h_l}
                }
              }
            }
            &
            \deduce{\{ - \mid \Delta_i \up B_i\}}{g_i}
        }
      }
    }
    \]
    If $f' = \lolliL\otR^+(\{f_j\},\blurLR(k),\{h_l\})$ and the $f_i$ use resources in $\Gamma'$, so $\Gamma = \vec{\Gamma'}$ and $\Gamma' = \vec{\Gamma_1},\Lambda,\vec{\Omega}$ for $\vec{\Gamma_1} \not= -$:
    \[\footnotesize \hspace{-2cm}
    \proofbox{
      \infer[\tr^+_{\up N}]{\vec{A} \lolli Q \mid \vec{\Gamma_0},\vec{\Delta} \up M \ot \vec{B}}{
        \infer[\focLR]{\vec{A} \lolli Q \mid \vec{\Gamma_0},\vec{\Gamma_1}, \Lambda, \vec{\Omega}, \up M' \ot \vec{C}}{
          \infer[\lolliL\otR^+]{\vec{A} \lolli Q \dn \vec{\Gamma_0}, \vec{\Gamma_1}, \Lambda, \vec{\Omega}, \dn M' \ot \vec{C}}{
            \deduce{\{ - \mid \Gamma_j\up A_j\}}{f_j}
            &
            \infer[\blurLR]{Q \dn \Lambda \dn M'}{
              \deduce{Q \mid \Lambda \up M'}{k}
            }
            &
            \deduce{\{ - \mid \Omega_l \up C_l\}}{h_l}
          }
      }
          \deduce{\{ - \mid \Delta_i \up B_i\}}{g_i}            
      }
    }
    \]
    \[\footnotesize \hspace{2cm}
    =
    \proofbox{
      \infer[\focR]{\vec{A} \lolli Q \mid \vec{\Gamma_0},\vec{\Delta} \up M \ot \vec{B}}{
        \infer[\otR^+]{\vec{A} \lolli Q \mid \vec{\Gamma_0},\vec{\Delta} \dn M \ot \vec{B}}{
          \infer[\blurR]{\vec{A} \lolli Q \mid \vec{\Gamma_0} \dn M}{
            \infer[\lolliR^+]{\vec{A} \lolli Q \mid \vec{\Gamma_0} \up^r M}{
              \infer[\focLR]{\vec{A} \lolli Q \mid \vec{\Gamma_0}, \vec{\Gamma_1}^r, \Lambda^r, \vec{\Omega}^r \up^r M' \ot \vec{C}}{
                \infer[\lolliL\otR^+]{\vec{A} \lolli Q \dn \vec{\Gamma_0}, \vec{\Gamma_1}^r, \Lambda^r, \vec{\Omega}^r \dn^r M' \ot \vec{C}}{
                  \deduce{\{ - \mid \Gamma_j\up A_j\}}{f_j}
                  &
                  \infer[\blurLR]{Q \dn \Lambda \dn M'}{
                    \deduce{Q \mid \Lambda \up M'}{k}
                  }
                  &
                  \deduce{\{ - \mid \Omega_l \up C_l\}}{h_l}
                }
              }
            }
          }
          &
          \deduce{\{ - \mid \Delta_i \up B_i\}}{g_i}            
        }
      }
    }
    \]
    \fbox{Case $f = \focR(f')$} with $f' : S_M \mid \Gamma,\Gamma' \dn P$:
    \[\footnotesize
    \proofbox{
      \infer[\tr_{\up P}^+]{S_M \mid \Gamma , \vec{\Delta} \up M \ot \vec{B}}{
        \infer[\focR]{S_M \mid \Gamma,\Gamma' \up P}{
          \deduce{S_M \mid \Gamma,\Gamma' \dn P}{f'}
        }
        &
        \deduce{\{ - \mid \Delta_i \up B_i\}}{g_i}
      }
    }
    =
    \proofbox{
      \infer[\focR]{S_M \mid \Gamma , \vec{\Delta} \up M \ot \vec{B}}{
        \infer[\otR^+]{S_M \mid \Gamma , \vec{\Delta} \dn M \ot \vec{B}}{
          \infer[\blurR]{S_M \mid \Gamma \dn M}{
            \infer[\lolliR^+]{S_M \mid \Gamma \up^r P}{
              \infer[\focR]{S_M \mid \Gamma,\Gamma'^r \up^r P}{
                \deduce{S_M \mid \Gamma,\Gamma' \dn P}{f'}
              }
            }
          }
          &
          \deduce{\{ - \mid \Delta_i \up B_i\}}{g_i}
        }
      }
    }
    \]
\end{proof}

\begin{proposition}\label{prop:adm2}
  The following rules are admissible:
  \[
%  \infer[\tr_{\up Q}^+]{S \mid^{\ell} \Gamma , \vec{\Delta} \up Q \ot \vec{B}}{
%      S \mid^{\ell} \Gamma \up Q
%      &
%      \{ - \mid \Delta_i \up B_i\}_{i>0}
%  }
%  \qquad
  \infer[\tr_{\up}]{S \mid \Gamma , \Delta \up A \ot B}{
      S \mid \Gamma \up A
      &
      - \mid \Delta \up B
%      & \Gamma' \not= -
  }
  \qquad
   \infer[\lleft_{\up}]{A \lolli B \mid \Gamma , \Delta \up C}{
     - \mid \Gamma \up A
     &
     B \mid \Delta \up C
   }    
   \qquad
   \infer[\ax_\up]{A \mid ~ \up A}{}
   \]
\end{proposition}
\begin{proof}
$\tr_\up$ can be proved by case analysis on $A$ using Lemma~\ref{lem:otR}. One can prove a lemma similar (in a way ``dual'') to Lemma~\ref{lem:otR} but involving $\lolliL$ instead of $\otR$, which allows to prove $\lleft_{\up}$. 
\end{proof}

Putting together Propositions \ref{prop:adm1} and \ref{prop:adm2}, it is possible to construct a function $\focus: S \mid \Gamma\vdash A\to S \mid \Gamma \up A$ sending each generating rule in (\ref{fig:seqcalc}) to its admissible counterpart (with subscript $\up$) in the focused calculus.

Moreover, the function $\focus$ sends $\circeq$-equivalent derivations to equal derivations, i.e. $f \circeq g$ implies $\focus(f) = \focus(g)$.


\bibliographystyle{splncs04}
\bibliography{multifocus}

\end{document}
